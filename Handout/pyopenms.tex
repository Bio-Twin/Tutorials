%!TEX root = handout.tex

\newpage
\section{An introduction to pyOpenMS}

\subsection{Introduction}
pyOpenMS provides Python bindings for a large part of the OpenMS library for mass spectrometry based proteomics. It thus provides access to a feature-rich, open-source algorithm library for mass-spectrometry based proteomics analysis. These Python bindings allow raw access to the data-structures and algorithms implemented in {OpenMS, specifically those for file access (mzXML, mzML, TraML, mzIdentML among others), basic signal processing (smoothing, filtering, de-isotoping and peak-picking) and complex data analysis (including label-free, SILAC, iTRAQ and SWATH analysis tools).\\

pyOpenMS is integrated into OpenMS starting from version 1.11. This tutorial is addressed to people already familiar with Python. If you are new to Python, we suggest to start with a Python tutorial (\url{http://en.wikibooks.org/wiki/Non-Programmer%27s_Tutorial_for_Python_2.6}).

\subsection{Installation}
\subsubsection{Windows}
\begin{enumerate}
  \item Install Python 2.7 (\url{http://www.python.org/download/})
  \item Install NumPy (\url{http://www.lfd.uci.edu/~gohlke/pythonlibs/#numpy})
  \item Install setuptools (\url{https://pypi.python.org/pypi/setuptools})
  \item On the command line:
    \begin{lstlisting}
    easy_install pyopenms
    \end{lstlisting}
\end{enumerate}

\subsubsection{Mac OS X 10.10}
\begin{enumerate}
  \item On the Terminal:
    \begin{lstlisting}
    sudo easy_install pyopenms
    \end{lstlisting}
\end{enumerate}

\subsubsection{Linux}
\begin{enumerate}
  \item Install Python 2.6 or 2.7 (Debian: python-dev, RedHat: python-devel)
  \item Install NumPy (Debian / RedHat: python-numpy)
  \item Install setuptools (Debian / RedHat: python-setuptools)
  \item On the Terminal:
    \begin{lstlisting}
    easy_install pyopenms
    \end{lstlisting}
\end{enumerate}

\subsection{Build instructions}
Instructions on how to build pyOpenMS can be found online (\url{http://ftp.mi.fu-berlin.de/OpenMS/release-documentation/html/pyOpenMS.html}).

\subsection{Your first pyOpenMS tool: pyOpenSwathFeatureXMLToTSV}
The first tool that you are going to re-implement is a TOPP tool called OpenSwathFeatureXMLToTSV. Take a look at the help of the tool:

\begin{lstlisting}
OpenSwathFeatureXMLToTSV --help

OpenSwathFeatureXMLToTSV -- Converts a featureXML to a mProphet tsv.
Version: 2.0.0 Apr 11 2015, 02:02:58, Revision: 66a7739

Usage:
  OpenSwathFeatureXMLToTSV <options>

Options (mandatory options marked with '*'):
  -in <files>*                     Input files separated by blank (valid formats: 'featureXML')
  -tr <file>*                      TraML transition file (valid formats: 'traML')
  -out <file>*                     Tsv output file (mProphet compatible) (valid formats: 'csv')
  -short_format                    Whether to write short (one peptide per line) or long format (one transition per line).
  -best_scoring_peptide <varname>  If only the best scoring feature per peptide should be printed, give the variable name
                                   
Common TOPP options:
  -ini <file>                      Use the given TOPP INI file
  -threads <n>                     Sets the number of threads allowed to be used by the TOPP tool (default: '1')
  -write_ini <file>                Writes the default configuration file
  --help                           Shows options
  --helphelp                       Shows all options (including advanced)
\end{lstlisting}

OpenSwathFeatureXMLToTSV converts a featureXML file to a tab-separated value text file. This example will teach you how to use pyOpenMS in combination with Python to implement such a tool very quickly.

\subsubsection{Basics}
The first task that your tool needs to be able to do is to read the parameters from the command line and provide a main routine. This is all standard Python and no pyOpenMS is needed so far:

\begin{lstlisting}
#!/usr/bin/env python
import sys

def main(options):

    # test parameter handling
    print options.infile, options.traml_in, options.outfile

def handle_args():
    import argparse

    usage = ""
    usage += "\nOpenSwathFeatureXMLToTSV −− Converts a featureXML to a mProphet tsv."

    parser = argparse.ArgumentParser(description = usage )
    parser.add_argument('-in', dest='infile', help = 'An input file containing features􏰀 featureXML]')
    parser.add_argument('-tr', dest='traml_in', help='An input file containing the transitions TraML]')
    parser.add_argument('-out', dest='outfile', help='Output mProphet TSV file [tsv]')

    args = parser.parse_args(sys.argv[1:])
    return args

if __name__ == '__main__':
    options = handle_args()
    main(options)
\end{lstlisting}

Execute this code in the example script

\begin{center}
\texttt{./pyOpenMS/OpenSwathFeatureXMLToTSV\_basics.py}
\end{center}

\begin{lstlisting}
python OpenSwathFeatureXMLToTSV_basics.py --help
usage: OpenSwathFeatureXMLToTSV_basics.py [-h] [-in INFILE] [-tr TRAML_IN]
                                          [-out OUTFILE]

OpenSwathFeatureXMLToTSV −− Converts a featureXML to a mProphet tsv.

optional arguments:
  -h, --help    show this help message and exit
  -in INFILE    An input file containing features [featureXML]
  -tr TRAML_IN  An input file containing the transitions [TraML]
  -out OUTFILE  Output mProphet TSV file [tsv]
\end{lstlisting}

\begin{lstlisting}
python OpenSwathFeatureXMLToTSV_basics.py -in data/example.featureXML -tr assay/OpenSWATH_SGS_AssayLibrary.TraML -out example.tsv  
data/example.featureXML assay/OpenSWATH_SGS_AssayLibrary.TraML example.tsv
\end{lstlisting}

The parameters are being read from the command line by the function handle\_args() and given to the main() function of the script, which prints the different variables.

\subsubsection{Loading data structures with pyOpenMS}
Now we're going to import the pyOpenMS module with \texttt{import pyopenms} in the header of the script and load the featureXML:

\begin{lstlisting}
#!/usr/bin/env python
import pyopenms
import sys

def main(options):
    #  load featureXML
    features = pyopenms.FeatureMap()
    fh = pyopenms.FileHandler()
    fh.loadFeatures(options.infile, features)
    keys = []
    features[0].getKeys(keys)
    print keys

def handle_args():
    import argparse

    usage = ""
    usage += "\nOpenSwathFeatureXMLToTSV −− Converts a featureXML to a mProphet tsv."

    parser = argparse.ArgumentParser(description = usage )
    parser.add_argument('-in', dest='infile', help = 'An input file containing features [featureXML]')
    parser.add_argument('-tr', dest='traml_in', help='An input file containing the transitions [TraML]')
    parser.add_argument('-out', dest='outfile', help='Output mProphet TSV file [tsv]')

    args = parser.parse_args(sys.argv[1:])
    return args

if __name__ == '__main__':
    options = handle_args()
    main(options)
\end{lstlisting}

The function pyopenms.FeatureMap() initializes an OpenMS FeatureMap data structure. The function pyopenms.FileHandler() prepares a filehandler with the variable name fh and fh.loadFeatures(options.infile, features) takes the filename and imports the featureXML into the FeatureMap data structure.

In the next step, we're accessing the keys using the function getKeys() and printing them to stdout:

\begin{lstlisting}
python OpenSwathFeatureXMLToTSV_datastructures1.py -in data/example.featureXML -tr assay/OpenSWATH_SGS_AssayLibrary.TraML -out example.tsv
['PeptideRef', 'leftWidth', 'rightWidth', 'total_xic', 'peak_apices_sum', 'var_xcorr_coelution', 'var_xcorr_coelution_weighted ', 'var_xcorr_shape', 'var_xcorr_shape_weighted', 'var_library_corr', 'var_library_rmsd', 'var_library_manhattan', 'var_library_dotprod', 'delta_rt', 'assay_rt', 'norm_RT', 'rt_score', 'var_norm_rt_score', 'var_intensity_score', 'nr_peaks', 'sn_ratio', 'var_log_sn_score', 'var_elution_model_fit_score', 'xx_lda_prelim_score', 'var_isotope_correlation_score', 'var_isotope_overlap_score', 'var_massdev_score', 'var_massdev_score_weighted', 'var_bseries_score', 'var_yseries_score', 'var_dotprod_score', 'var_manhatt_score', 'main_var_xx_swath_prelim_score', 'PrecursorMZ', 'xx_swath_prelim_score']
\end{lstlisting}

In the next task, please load the TraML into an OpenMS TargetedExperiment data structure, analogously to the featureXML. You might want to consult the pyOpenMS manual (\url{http://proteomics.ethz.ch/pyOpenMS_Manual.pdf}), which provides an overview of all functionality. If you have trouble reading the TraML, search for TraMLFile(). If you can't solve the task, take a look at OpenSwathFeatureXMLToTSV\_datastructures2.py

\subsubsection{Converting data in the featureXML to a TSV}
Now that all data structures are populated, we need to access the data using the provided API and store it in something that is directly accessible from Python. We prepared two functions for you: get\_header() \& convert\_to\_row():

\begin{lstlisting}
def get_header(features):
    keys = []
    features[0].getKeys(keys)
    header = [
        "transition_group_id",
        "run_id",
        "filename",
        "RT",
        "id",
        "Sequence" ,
        "FullPeptideName",
        "Charge",
        "m/z",
        "Intensity",
        "ProteinName",
        "decoy"]
    header.extend(keys)
    return header
\end{lstlisting}

get\_header() takes as input a FeatureMap and uses the getKeys() function that you have seen before to extend a predefined header list based on the contents of the FeatureMap. The return variable is a native Python list.

\begin{lstlisting}
def convert_to_row(first, targ, run_id, keys, filename):
    peptide_ref = first.getMetaValue("PeptideRef")
    pep = targ.getPeptideByRef(peptide_ref)
    full_peptide_name = "NA"
    if (pep.metaValueExists("full_peptide_name")):
        full_peptide_name = pep.getMetaValue("full_peptide_name")

    decoy = "0"
    peptidetransitions = [t for t in targ.getTransitions() if t.getPeptideRef() == peptide_ref]
    if len(peptidetransitions) > 0:
        if peptidetransitions[0].getDecoyTransitionType() == pyopenms.DecoyTransitionType().DECOY:
            decoy = "1"
        elif peptidetransitions[0].getDecoyTransitionType() == pyopenms.DecoyTransitionType().TARGET:
            decoy = "0"

    protein_name = "NA"
    if len(pep.protein_refs) > 0:
        protein_name = pep.protein_refs[0]

    row = [
        first.getMetaValue("PeptideRef"),
        run_id,
        filename,
        first.getRT(),
        first.getUniqueId(),
        pep.sequence,
        full_peptide_name,
        pep.getChargeState(),
        first.getMetaValue("PrecursorMZ"),
        first.getIntensity(),
        protein_name,
        decoy
    ]

    for k in keys:
        row.append(first.getMetaValue(k))

    return row
\end{lstlisting}

convert\_to\_row() is a bit more complicated and takes as first input a Feature OpenMS class. From this, we access stored values using the provided functions (getRT(), getUniqueId(), etc). It further takes a TargetedExperiment to access information from the TraML with the provided routines. This data is then stored in a standard Python list with the variable name row and returned.

\subsubsection{Putting things together}

Now put these two functions into the header of 

\begin{center}
\texttt{OpenSwathFeatureXMLToTSV\_datastructures2.py}. 
\end{center}

Your final goal is to implement the conversion functionality into the main function using get\_header() \& convert\_to\_row() and to write a TSV using the standard csv module from Python \url{http://docs.python.org/2/library/csv.html}. 
Compare the results with \texttt{./result/example.tsv}. Are the results identical? Congratulations to your first pyOpenMS tool!\\

Hint: If you struggle at any point, take a look at OpenSwathFeatureXMLToTSV\_solution.py.

\subsubsection{Bonus task}

\begin{task}
Implement all other 184 TOPP tools using pyOpenMS.
\end{task}
