%!TEX root = handout.tex

\section{General remarks}
\label{General remarks}

\begin{itemize}
\item This handout will guide you through the tutorial ``Fundamentals of Proteome Bioinformatics Revisited''.
\item In this hands-on tutorial session, you will become familiar with some of the tutorial topics with hands-on and homework and the basic functionalities of KNIME and OpenMS/TOPP, as well as learn how to use a selection of TOPP tools used in the tutorial workflows.
\item OpenMS~\cite{Sturm2008} is a versatile open-source library for mass spectrometry data analysis. Based on this library, we offer a collection
      of command-line tools ready to be used by end users. These so-called TOPP tools (short for ``The OpenMS Proteomics Pipeline'')~\cite{Kohlbacher2007} can be understood as
      small building blocks of arbitrary complex data analysis workflows.
\item In order to facilitate workflow construction, OpenMS was integrated into KNIME~\cite{KNIME}, the Konstanz
			Information Miner, an open-source integration platform providing a powerful and flexible workflow system
			combined with advanced data analytics, visualisation, and report capabilities. Raw MS data as well as the
			results of data processing using TOPP can be visualized using TOPPView~\cite{Sturm2009}.
\item All data referenced in this tutorial can be found in the \directory{Example\_Data} folder that came with this tutorial.
%\item Installation packages of the new OpenMS 1.11 release for Windows (32 and 64 bit) and Mac OS X (10.7 and 10.8) as well as
%      all data required for this tutorial session can be found on the USB sticks provided by us.
\end{itemize}


\setcounter{equation}{0}
